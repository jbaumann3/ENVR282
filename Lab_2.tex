% Options for packages loaded elsewhere
\PassOptionsToPackage{unicode}{hyperref}
\PassOptionsToPackage{hyphens}{url}
\PassOptionsToPackage{dvipsnames,svgnames,x11names}{xcolor}
%
\documentclass[
  letterpaper,
  DIV=11,
  numbers=noendperiod]{scrartcl}

\usepackage{amsmath,amssymb}
\usepackage{iftex}
\ifPDFTeX
  \usepackage[T1]{fontenc}
  \usepackage[utf8]{inputenc}
  \usepackage{textcomp} % provide euro and other symbols
\else % if luatex or xetex
  \usepackage{unicode-math}
  \defaultfontfeatures{Scale=MatchLowercase}
  \defaultfontfeatures[\rmfamily]{Ligatures=TeX,Scale=1}
\fi
\usepackage{lmodern}
\ifPDFTeX\else  
    % xetex/luatex font selection
\fi
% Use upquote if available, for straight quotes in verbatim environments
\IfFileExists{upquote.sty}{\usepackage{upquote}}{}
\IfFileExists{microtype.sty}{% use microtype if available
  \usepackage[]{microtype}
  \UseMicrotypeSet[protrusion]{basicmath} % disable protrusion for tt fonts
}{}
\makeatletter
\@ifundefined{KOMAClassName}{% if non-KOMA class
  \IfFileExists{parskip.sty}{%
    \usepackage{parskip}
  }{% else
    \setlength{\parindent}{0pt}
    \setlength{\parskip}{6pt plus 2pt minus 1pt}}
}{% if KOMA class
  \KOMAoptions{parskip=half}}
\makeatother
\usepackage{xcolor}
\setlength{\emergencystretch}{3em} % prevent overfull lines
\setcounter{secnumdepth}{5}
% Make \paragraph and \subparagraph free-standing
\ifx\paragraph\undefined\else
  \let\oldparagraph\paragraph
  \renewcommand{\paragraph}[1]{\oldparagraph{#1}\mbox{}}
\fi
\ifx\subparagraph\undefined\else
  \let\oldsubparagraph\subparagraph
  \renewcommand{\subparagraph}[1]{\oldsubparagraph{#1}\mbox{}}
\fi

\usepackage{color}
\usepackage{fancyvrb}
\newcommand{\VerbBar}{|}
\newcommand{\VERB}{\Verb[commandchars=\\\{\}]}
\DefineVerbatimEnvironment{Highlighting}{Verbatim}{commandchars=\\\{\}}
% Add ',fontsize=\small' for more characters per line
\usepackage{framed}
\definecolor{shadecolor}{RGB}{241,243,245}
\newenvironment{Shaded}{\begin{snugshade}}{\end{snugshade}}
\newcommand{\AlertTok}[1]{\textcolor[rgb]{0.68,0.00,0.00}{#1}}
\newcommand{\AnnotationTok}[1]{\textcolor[rgb]{0.37,0.37,0.37}{#1}}
\newcommand{\AttributeTok}[1]{\textcolor[rgb]{0.40,0.45,0.13}{#1}}
\newcommand{\BaseNTok}[1]{\textcolor[rgb]{0.68,0.00,0.00}{#1}}
\newcommand{\BuiltInTok}[1]{\textcolor[rgb]{0.00,0.23,0.31}{#1}}
\newcommand{\CharTok}[1]{\textcolor[rgb]{0.13,0.47,0.30}{#1}}
\newcommand{\CommentTok}[1]{\textcolor[rgb]{0.37,0.37,0.37}{#1}}
\newcommand{\CommentVarTok}[1]{\textcolor[rgb]{0.37,0.37,0.37}{\textit{#1}}}
\newcommand{\ConstantTok}[1]{\textcolor[rgb]{0.56,0.35,0.01}{#1}}
\newcommand{\ControlFlowTok}[1]{\textcolor[rgb]{0.00,0.23,0.31}{#1}}
\newcommand{\DataTypeTok}[1]{\textcolor[rgb]{0.68,0.00,0.00}{#1}}
\newcommand{\DecValTok}[1]{\textcolor[rgb]{0.68,0.00,0.00}{#1}}
\newcommand{\DocumentationTok}[1]{\textcolor[rgb]{0.37,0.37,0.37}{\textit{#1}}}
\newcommand{\ErrorTok}[1]{\textcolor[rgb]{0.68,0.00,0.00}{#1}}
\newcommand{\ExtensionTok}[1]{\textcolor[rgb]{0.00,0.23,0.31}{#1}}
\newcommand{\FloatTok}[1]{\textcolor[rgb]{0.68,0.00,0.00}{#1}}
\newcommand{\FunctionTok}[1]{\textcolor[rgb]{0.28,0.35,0.67}{#1}}
\newcommand{\ImportTok}[1]{\textcolor[rgb]{0.00,0.46,0.62}{#1}}
\newcommand{\InformationTok}[1]{\textcolor[rgb]{0.37,0.37,0.37}{#1}}
\newcommand{\KeywordTok}[1]{\textcolor[rgb]{0.00,0.23,0.31}{#1}}
\newcommand{\NormalTok}[1]{\textcolor[rgb]{0.00,0.23,0.31}{#1}}
\newcommand{\OperatorTok}[1]{\textcolor[rgb]{0.37,0.37,0.37}{#1}}
\newcommand{\OtherTok}[1]{\textcolor[rgb]{0.00,0.23,0.31}{#1}}
\newcommand{\PreprocessorTok}[1]{\textcolor[rgb]{0.68,0.00,0.00}{#1}}
\newcommand{\RegionMarkerTok}[1]{\textcolor[rgb]{0.00,0.23,0.31}{#1}}
\newcommand{\SpecialCharTok}[1]{\textcolor[rgb]{0.37,0.37,0.37}{#1}}
\newcommand{\SpecialStringTok}[1]{\textcolor[rgb]{0.13,0.47,0.30}{#1}}
\newcommand{\StringTok}[1]{\textcolor[rgb]{0.13,0.47,0.30}{#1}}
\newcommand{\VariableTok}[1]{\textcolor[rgb]{0.07,0.07,0.07}{#1}}
\newcommand{\VerbatimStringTok}[1]{\textcolor[rgb]{0.13,0.47,0.30}{#1}}
\newcommand{\WarningTok}[1]{\textcolor[rgb]{0.37,0.37,0.37}{\textit{#1}}}

\providecommand{\tightlist}{%
  \setlength{\itemsep}{0pt}\setlength{\parskip}{0pt}}\usepackage{longtable,booktabs,array}
\usepackage{calc} % for calculating minipage widths
% Correct order of tables after \paragraph or \subparagraph
\usepackage{etoolbox}
\makeatletter
\patchcmd\longtable{\par}{\if@noskipsec\mbox{}\fi\par}{}{}
\makeatother
% Allow footnotes in longtable head/foot
\IfFileExists{footnotehyper.sty}{\usepackage{footnotehyper}}{\usepackage{footnote}}
\makesavenoteenv{longtable}
\usepackage{graphicx}
\makeatletter
\def\maxwidth{\ifdim\Gin@nat@width>\linewidth\linewidth\else\Gin@nat@width\fi}
\def\maxheight{\ifdim\Gin@nat@height>\textheight\textheight\else\Gin@nat@height\fi}
\makeatother
% Scale images if necessary, so that they will not overflow the page
% margins by default, and it is still possible to overwrite the defaults
% using explicit options in \includegraphics[width, height, ...]{}
\setkeys{Gin}{width=\maxwidth,height=\maxheight,keepaspectratio}
% Set default figure placement to htbp
\makeatletter
\def\fps@figure{htbp}
\makeatother

\KOMAoption{captions}{tableheading}
\makeatletter
\makeatother
\makeatletter
\makeatother
\makeatletter
\@ifpackageloaded{caption}{}{\usepackage{caption}}
\AtBeginDocument{%
\ifdefined\contentsname
  \renewcommand*\contentsname{Table of contents}
\else
  \newcommand\contentsname{Table of contents}
\fi
\ifdefined\listfigurename
  \renewcommand*\listfigurename{List of Figures}
\else
  \newcommand\listfigurename{List of Figures}
\fi
\ifdefined\listtablename
  \renewcommand*\listtablename{List of Tables}
\else
  \newcommand\listtablename{List of Tables}
\fi
\ifdefined\figurename
  \renewcommand*\figurename{Figure}
\else
  \newcommand\figurename{Figure}
\fi
\ifdefined\tablename
  \renewcommand*\tablename{Table}
\else
  \newcommand\tablename{Table}
\fi
}
\@ifpackageloaded{float}{}{\usepackage{float}}
\floatstyle{ruled}
\@ifundefined{c@chapter}{\newfloat{codelisting}{h}{lop}}{\newfloat{codelisting}{h}{lop}[chapter]}
\floatname{codelisting}{Listing}
\newcommand*\listoflistings{\listof{codelisting}{List of Listings}}
\makeatother
\makeatletter
\@ifpackageloaded{caption}{}{\usepackage{caption}}
\@ifpackageloaded{subcaption}{}{\usepackage{subcaption}}
\makeatother
\makeatletter
\@ifpackageloaded{tcolorbox}{}{\usepackage[skins,breakable]{tcolorbox}}
\makeatother
\makeatletter
\@ifundefined{shadecolor}{\definecolor{shadecolor}{rgb}{.97, .97, .97}}
\makeatother
\makeatletter
\makeatother
\makeatletter
\makeatother
\ifLuaTeX
  \usepackage{selnolig}  % disable illegal ligatures
\fi
\IfFileExists{bookmark.sty}{\usepackage{bookmark}}{\usepackage{hyperref}}
\IfFileExists{xurl.sty}{\usepackage{xurl}}{} % add URL line breaks if available
\urlstyle{same} % disable monospaced font for URLs
\hypersetup{
  pdftitle={Lab 2: Intro to graphing in ggplot},
  pdfauthor={Justin Baumann},
  colorlinks=true,
  linkcolor={blue},
  filecolor={Maroon},
  citecolor={Blue},
  urlcolor={Blue},
  pdfcreator={LaTeX via pandoc}}

\title{Lab 2: Intro to graphing in ggplot}
\author{Justin Baumann}
\date{}

\begin{document}
\maketitle
\ifdefined\Shaded\renewenvironment{Shaded}{\begin{tcolorbox}[borderline west={3pt}{0pt}{shadecolor}, frame hidden, breakable, interior hidden, boxrule=0pt, enhanced, sharp corners]}{\end{tcolorbox}}\fi

\renewcommand*\contentsname{Table of contents}
{
\hypersetup{linkcolor=}
\setcounter{tocdepth}{3}
\tableofcontents
}
\hypertarget{tutorials-and-resources-for-graphs-in-ggplot}{%
\section{Tutorials and Resources for graphs in
ggplot}\label{tutorials-and-resources-for-graphs-in-ggplot}}

\href{http://www.cookbook-r.com/graphs}{Basics of ggplot}

\href{https://cran.r-project.org/web/packages/ggsci/vignettes/ggsci.html}{Colors
with ggsci}

\href{https://github.com/thomasp85/patchwork}{Many plots, 1 page w/
Patchwork}

\hypertarget{load-packages-we-need}{%
\section{\texorpdfstring{\textbf{1.) Load packages we
need}}{1.) Load packages we need}}\label{load-packages-we-need}}

Making nice looking graphs is a key feature of R and of data science in
general. The best way to do this in R is through use of the ggplot2
package. This package is the most user friendly and flexible way to make
nice plots in R. Notably, ggplot2 is a package that is contained within
the tidyverse package, which is more of a style of R usage than a
package. So, let's load tidyverse and a few other useful packages for
today.

\begin{Shaded}
\begin{Highlighting}[]
\CommentTok{\#Load packages}
\FunctionTok{library}\NormalTok{(tidyverse)}
\FunctionTok{library}\NormalTok{(ggsci) }\CommentTok{\#for easy color scales}
\FunctionTok{library}\NormalTok{(patchwork) }\CommentTok{\#to make multi{-}panel plots }
\FunctionTok{library}\NormalTok{(palmerpenguins) }\CommentTok{\# our fave penguin friends :)}
\end{Highlighting}
\end{Shaded}

\hypertarget{what-makes-a-good-graph-vs-a-bad-graph}{%
\section{\texorpdfstring{\textbf{2.) What makes a good graph vs a bad
graph?}}{2.) What makes a good graph vs a bad graph?}}\label{what-makes-a-good-graph-vs-a-bad-graph}}

Take a look at some graphs of data for your field of interest. You may
have a look at papers you have recently read or graphs you find in
textbooks or assignments. Consider what you like or don't like about
these graphs. What looks good and/or makes a graph easy to interpret?
What doesn't? Making figures is both an art and a science.

To learn more about what makes graphs good (or bad), read Chapter 1 of
Kieran Healy's online data visualization book --\textgreater{}
\href{https://socviz.co/lookatdata.html\#lookatdata}{What makes figures
bad?}\\
\strut \\
To continue your learning, have a look at this more detailed data
visualization book by Claus Wilke
\href{https://clauswilke.com/dataviz/index.html}{Fundamentals of Data
Visualization}\\

\hypertarget{ggplot-basics}{%
\section{\texorpdfstring{\textbf{3.) ggplot
basics}}{3.) ggplot basics}}\label{ggplot-basics}}

\subsection{\texorpdfstring{\textbf{Introduction}}{Introduction}}

ggplot2 is the preferred graphics package for most R users. It allows
users to build a graph piece by piece from your own data through mapping
of aesthetics. It is much easier to make pretty (publication and
presentation quality) plots with ggplot2 than it is with the base plot
function in R. If you prefer base plot() that is ok. You can use
whatever you'd like but when we talk about graphs we will be using the
language of ggplot.

Attached here are the Tidyverse Cheat Sheets for ggplot2

\includegraphics{images/ggplot cheat sheet 2.png}\\

\includegraphics{images/ggplot cheat sheet 2.png}\\

\subsection{\texorpdfstring{\textbf{ggplot()}}{ggplot()}}

The ggplot() function is the base of the ggplot2 package. Using it
creates the space that we use to build a graph. If we run just the
ggplot() function we will get a gray rectangle. This is the space (and
background) of our plot!

\begin{Shaded}
\begin{Highlighting}[]
\FunctionTok{ggplot}\NormalTok{()}
\end{Highlighting}
\end{Shaded}

\begin{figure}[H]

{\centering \includegraphics{Lab_2_files/figure-pdf/unnamed-chunk-2-1.pdf}

}

\end{figure}

To build a plot on the background, we must add to the ggplot call.
First, we need to tell it what data to use. Next, we need to tell it
where in the data frame to pull data from to build the axes and data
points. The part of the ggplot() function we use to build a graph is
called aes() or aesthetics.

Here is an example using penguins: I am telling ggplot that the data we
are using is `penguins' and then defining the x and y axis in the aes()
call with column names from penguins

\begin{Shaded}
\begin{Highlighting}[]
\FunctionTok{head}\NormalTok{(penguins)}
\end{Highlighting}
\end{Shaded}

\begin{verbatim}
# A tibble: 6 x 8
  species island    bill_length_mm bill_depth_mm flipper_length_mm body_mass_g
  <fct>   <fct>              <dbl>         <dbl>             <int>       <int>
1 Adelie  Torgersen           39.1          18.7               181        3750
2 Adelie  Torgersen           39.5          17.4               186        3800
3 Adelie  Torgersen           40.3          18                 195        3250
4 Adelie  Torgersen           NA            NA                  NA          NA
5 Adelie  Torgersen           36.7          19.3               193        3450
6 Adelie  Torgersen           39.3          20.6               190        3650
# i 2 more variables: sex <fct>, year <int>
\end{verbatim}

\begin{Shaded}
\begin{Highlighting}[]
\FunctionTok{ggplot}\NormalTok{(}\AttributeTok{data=}\NormalTok{penguins, }\FunctionTok{aes}\NormalTok{(}\AttributeTok{x=}\NormalTok{species, }\AttributeTok{y=}\NormalTok{ bill\_length\_mm)) }
\end{Highlighting}
\end{Shaded}

\begin{figure}[H]

{\centering \includegraphics{Lab_2_files/figure-pdf/unnamed-chunk-3-1.pdf}

}

\end{figure}

Like anything in R, we can give our plot a name and call it later

\begin{Shaded}
\begin{Highlighting}[]
\NormalTok{plot1}\OtherTok{\textless{}{-}}\FunctionTok{ggplot}\NormalTok{(}\AttributeTok{data=}\NormalTok{penguins, }\FunctionTok{aes}\NormalTok{(}\AttributeTok{x=}\NormalTok{species, }\AttributeTok{y=}\NormalTok{ bill\_length\_mm)) }

\NormalTok{plot1}
\end{Highlighting}
\end{Shaded}

\begin{figure}[H]

{\centering \includegraphics{Lab_2_files/figure-pdf/unnamed-chunk-4-1.pdf}

}

\end{figure}

This is incredibly useful in ggplot as we can essentially add pieces to
make a more complete graph

\begin{Shaded}
\begin{Highlighting}[]
\NormalTok{plot1}\SpecialCharTok{+}
  \FunctionTok{geom\_boxplot}\NormalTok{()}\SpecialCharTok{+}
  \FunctionTok{geom\_point}\NormalTok{()}\SpecialCharTok{+}
  \FunctionTok{theme\_bw}\NormalTok{()}
\end{Highlighting}
\end{Shaded}

\begin{verbatim}
Warning: Removed 2 rows containing non-finite outside the scale range
(`stat_boxplot()`).
\end{verbatim}

\begin{verbatim}
Warning: Removed 2 rows containing missing values or values outside the scale range
(`geom_point()`).
\end{verbatim}

\begin{figure}[H]

{\centering \includegraphics{Lab_2_files/figure-pdf/unnamed-chunk-5-1.pdf}

}

\end{figure}

Before we get too excited about making perfect graphs, let's take a look
at the types of graphs we have available to us\ldots{}

\subsection{\texorpdfstring{\textbf{histogram}}{histogram}}

Histograms are used to explore the frequency distribution of a single
variable. We can check for normality (a bell curve) using this feature.
We can also look for means, skewed data, and other trends.

\begin{Shaded}
\begin{Highlighting}[]
\FunctionTok{ggplot}\NormalTok{(}\AttributeTok{data=}\NormalTok{penguins, }\FunctionTok{aes}\NormalTok{(bill\_length\_mm))}\SpecialCharTok{+}
  \FunctionTok{geom\_histogram}\NormalTok{()}
\end{Highlighting}
\end{Shaded}

\begin{verbatim}
`stat_bin()` using `bins = 30`. Pick better value with `binwidth`.
\end{verbatim}

\begin{verbatim}
Warning: Removed 2 rows containing non-finite outside the scale range
(`stat_bin()`).
\end{verbatim}

\begin{figure}[H]

{\centering \includegraphics{Lab_2_files/figure-pdf/unnamed-chunk-6-1.pdf}

}

\end{figure}

Within geom\_histogram we can use bin\_width to change the width of our
x-axis groupings.

\begin{Shaded}
\begin{Highlighting}[]
\FunctionTok{ggplot}\NormalTok{(}\AttributeTok{data=}\NormalTok{penguins, }\FunctionTok{aes}\NormalTok{(bill\_length\_mm))}\SpecialCharTok{+}
  \FunctionTok{geom\_histogram}\NormalTok{(}\AttributeTok{binwidth=}\DecValTok{5}\NormalTok{)}
\end{Highlighting}
\end{Shaded}

\begin{verbatim}
Warning: Removed 2 rows containing non-finite outside the scale range
(`stat_bin()`).
\end{verbatim}

\begin{figure}[H]

{\centering \includegraphics{Lab_2_files/figure-pdf/unnamed-chunk-7-1.pdf}

}

\end{figure}

\subsection{\texorpdfstring{\textbf{boxplot}}{boxplot}}

A boxplot is a really useful plot to assess median and range of data. It
can also identify outliers! The defaults for a boxplot in ggplot produce
a median and interquartile range (IQR). The 1st quartile is the bottom
of the box and the 3rd quartile is the top. The whiskers show the spread
of the data where the ends of the whiskers represent the data points
that are the furthest from the median in either direction. Notably, if a
data point is 1.5 * IQR from the box (either the 1st or 3rd quartile) it
is an outlier. Outliers are excluded from whiskers and are presented as
points. There

Here's an example

\begin{Shaded}
\begin{Highlighting}[]
\FunctionTok{ggplot}\NormalTok{(}\AttributeTok{data=}\NormalTok{penguins, }\FunctionTok{aes}\NormalTok{(}\AttributeTok{x=}\NormalTok{species, }\AttributeTok{y=}\NormalTok{ bill\_length\_mm)) }\SpecialCharTok{+}
  \FunctionTok{geom\_boxplot}\NormalTok{()}
\end{Highlighting}
\end{Shaded}

\begin{verbatim}
Warning: Removed 2 rows containing non-finite outside the scale range
(`stat_boxplot()`).
\end{verbatim}

\begin{figure}[H]

{\centering \includegraphics{Lab_2_files/figure-pdf/unnamed-chunk-8-1.pdf}

}

\end{figure}

We can use geom\_violin to combine boxplot with a density plot (similar
to a histogram) Here we can see the distribution of values within bill
length by species.

\begin{Shaded}
\begin{Highlighting}[]
\FunctionTok{ggplot}\NormalTok{(}\AttributeTok{data=}\NormalTok{penguins, }\FunctionTok{aes}\NormalTok{(}\AttributeTok{x=}\NormalTok{species, }\AttributeTok{y=}\NormalTok{ bill\_length\_mm)) }\SpecialCharTok{+}
  \CommentTok{\#geom\_boxplot()+}
  \FunctionTok{geom\_violin}\NormalTok{()}
\end{Highlighting}
\end{Shaded}

\begin{verbatim}
Warning: Removed 2 rows containing non-finite outside the scale range
(`stat_ydensity()`).
\end{verbatim}

\begin{figure}[H]

{\centering \includegraphics{Lab_2_files/figure-pdf/unnamed-chunk-9-1.pdf}

}

\end{figure}

\subsection{\texorpdfstring{\textbf{bar graph}}{bar graph}}

We can make bar graphs in ggplot using geom\_bar(). There are some
tricks to getting bar graphs to work exactly right, which I will try to
detail below. \textbf{NOTE} Bar graphs are very rarely useful. If we
want to show means, why not just use points + error bars? What does the
bar actually represent? There aren't that many cases where we really
need bar graphs. There are exceptions, like when we have a population
and we want to see the demographics of that population by count or
percentage (see example below)

Here is a simple bar chart.

\begin{Shaded}
\begin{Highlighting}[]
\FunctionTok{ggplot}\NormalTok{(}\AttributeTok{data=}\NormalTok{penguins, }\FunctionTok{aes}\NormalTok{(species)) }\SpecialCharTok{+}
  \FunctionTok{geom\_bar}\NormalTok{()}
\end{Highlighting}
\end{Shaded}

\begin{figure}[H]

{\centering \includegraphics{Lab_2_files/figure-pdf/unnamed-chunk-10-1.pdf}

}

\end{figure}

Here is a more elaborate boxplot that shows species breakdown by island!
Note that we use an aes() call within geom\_bar to define a fill. That
means fill by species, or add a color for each species.

\begin{Shaded}
\begin{Highlighting}[]
\FunctionTok{ggplot}\NormalTok{(}\AttributeTok{data=}\NormalTok{penguins, }\FunctionTok{aes}\NormalTok{(island)) }\SpecialCharTok{+}
  \FunctionTok{geom\_bar}\NormalTok{(}\FunctionTok{aes}\NormalTok{(}\AttributeTok{fill=}\NormalTok{species))}
\end{Highlighting}
\end{Shaded}

\begin{figure}[H]

{\centering \includegraphics{Lab_2_files/figure-pdf/unnamed-chunk-11-1.pdf}

}

\end{figure}

And here is that same plot with the bars unstacked. Instead of stacking,
we have used ``dodged'' each color to be its own bar.

\begin{Shaded}
\begin{Highlighting}[]
\FunctionTok{ggplot}\NormalTok{(}\AttributeTok{data=}\NormalTok{penguins, }\FunctionTok{aes}\NormalTok{(island)) }\SpecialCharTok{+}
  \FunctionTok{geom\_bar}\NormalTok{(}\FunctionTok{aes}\NormalTok{(}\AttributeTok{fill=}\NormalTok{species), }\AttributeTok{position=} \FunctionTok{position\_dodge}\NormalTok{())}
\end{Highlighting}
\end{Shaded}

\begin{figure}[H]

{\centering \includegraphics{Lab_2_files/figure-pdf/unnamed-chunk-12-1.pdf}

}

\end{figure}

We learned when the best (only) times to use bar graphs are. Do you
remember what those were? Are the examples above representative of that?

\subsection{\texorpdfstring{\textbf{line graph}}{line graph}}

A line graph can be extremely useful, especially if we are looking at
time series data or rates!

Here is an example of CO2 uptake vs concentration in plants. Each color
represents a different plant. NOTE: the dataset called `CO2' is built
into R, so we can just use it without loading anything :)

\begin{Shaded}
\begin{Highlighting}[]
\FunctionTok{head}\NormalTok{(CO2)}
\end{Highlighting}
\end{Shaded}

\begin{verbatim}
  Plant   Type  Treatment conc uptake
1   Qn1 Quebec nonchilled   95   16.0
2   Qn1 Quebec nonchilled  175   30.4
3   Qn1 Quebec nonchilled  250   34.8
4   Qn1 Quebec nonchilled  350   37.2
5   Qn1 Quebec nonchilled  500   35.3
6   Qn1 Quebec nonchilled  675   39.2
\end{verbatim}

\begin{Shaded}
\begin{Highlighting}[]
\FunctionTok{ggplot}\NormalTok{(}\AttributeTok{data=}\NormalTok{CO2, }\FunctionTok{aes}\NormalTok{(}\AttributeTok{x=}\NormalTok{conc, }\AttributeTok{y=}\NormalTok{ uptake, }\AttributeTok{color=}\NormalTok{Plant)) }\SpecialCharTok{+}
  \FunctionTok{geom\_line}\NormalTok{()}
\end{Highlighting}
\end{Shaded}

\begin{figure}[H]

{\centering \includegraphics{Lab_2_files/figure-pdf/unnamed-chunk-13-1.pdf}

}

\end{figure}

We can change the aesthetics of the lines using color, linetype, size,
etc. Here I am changing the linetype based on the plant species and
increasing the size of ALL lines to 2. This is a good example of how
aes() works. Anything within the aes() call is conditional. That means,
I give it a name (such as a column or variable name) and it changes
based on that column or variable. To change an aesthetic across all
lines, points, etc, I just put the code outside of the aes(). As I did
for size. That makes the size of ALL lines = 2.

\begin{Shaded}
\begin{Highlighting}[]
\FunctionTok{ggplot}\NormalTok{(}\AttributeTok{data=}\NormalTok{CO2, }\FunctionTok{aes}\NormalTok{(}\AttributeTok{x=}\NormalTok{conc, }\AttributeTok{y=}\NormalTok{ uptake, }\AttributeTok{color=}\NormalTok{Plant)) }\SpecialCharTok{+}
  \FunctionTok{geom\_line}\NormalTok{(}\FunctionTok{aes}\NormalTok{(}\AttributeTok{linetype=}\NormalTok{Plant),}\AttributeTok{size=}\DecValTok{2}\NormalTok{)}
\end{Highlighting}
\end{Shaded}

\begin{verbatim}
Warning: Using `size` aesthetic for lines was deprecated in ggplot2 3.4.0.
i Please use `linewidth` instead.
\end{verbatim}

\begin{figure}[H]

{\centering \includegraphics{Lab_2_files/figure-pdf/unnamed-chunk-14-1.pdf}

}

\end{figure}

\subsection{\texorpdfstring{\textbf{scatter plot}}{scatter plot}}

The scatter plot is probably the most commonly used graphical tool in
ggplot. It is based on the geom\_point() function

\begin{Shaded}
\begin{Highlighting}[]
\FunctionTok{ggplot}\NormalTok{(}\AttributeTok{data=}\NormalTok{penguins, }\FunctionTok{aes}\NormalTok{(}\AttributeTok{x=}\NormalTok{species, }\AttributeTok{y=}\NormalTok{ bill\_length\_mm)) }\SpecialCharTok{+}
  \FunctionTok{geom\_point}\NormalTok{()}
\end{Highlighting}
\end{Shaded}

\begin{verbatim}
Warning: Removed 2 rows containing missing values or values outside the scale range
(`geom_point()`).
\end{verbatim}

\begin{figure}[H]

{\centering \includegraphics{Lab_2_files/figure-pdf/unnamed-chunk-15-1.pdf}

}

\end{figure}

Importantly, we can use the data= and aes() calls within geom\_point()
or any other geom instead of within ggplot() if needed. Why might this
be important?

\begin{Shaded}
\begin{Highlighting}[]
\FunctionTok{ggplot}\NormalTok{() }\SpecialCharTok{+}
  \FunctionTok{geom\_point}\NormalTok{(}\AttributeTok{data=}\NormalTok{penguins, }\FunctionTok{aes}\NormalTok{(}\AttributeTok{x=}\NormalTok{species, }\AttributeTok{y=}\NormalTok{ bill\_length\_mm))}
\end{Highlighting}
\end{Shaded}

\begin{verbatim}
Warning: Removed 2 rows containing missing values or values outside the scale range
(`geom_point()`).
\end{verbatim}

\begin{figure}[H]

{\centering \includegraphics{Lab_2_files/figure-pdf/unnamed-chunk-16-1.pdf}

}

\end{figure}

Sometimes we don't want to plot all of our points on the same vertical
line. If that is the case, we can use geom\_jitter()

\begin{Shaded}
\begin{Highlighting}[]
\FunctionTok{ggplot}\NormalTok{(}\AttributeTok{data=}\NormalTok{penguins, }\FunctionTok{aes}\NormalTok{(}\AttributeTok{x=}\NormalTok{species, }\AttributeTok{y=}\NormalTok{ bill\_length\_mm)) }\SpecialCharTok{+}
  \FunctionTok{geom\_jitter}\NormalTok{()}
\end{Highlighting}
\end{Shaded}

\begin{verbatim}
Warning: Removed 2 rows containing missing values or values outside the scale range
(`geom_point()`).
\end{verbatim}

\begin{figure}[H]

{\centering \includegraphics{Lab_2_files/figure-pdf/unnamed-chunk-17-1.pdf}

}

\end{figure}

\subsection{\texorpdfstring{\textbf{Adding error
bars}}{Adding error bars}}

We often want to present means and error in our visualizations. This can
be done through the use of geom\_boxplot() or through combining
geom\_point() with geom\_errorbar()

Here is an example of the later\ldots{}

\begin{Shaded}
\begin{Highlighting}[]
\CommentTok{\#First, we need to calculate a mean bill length for our penguins by species and island}
\NormalTok{sumpens}\OtherTok{\textless{}{-}}\NormalTok{ penguins }\SpecialCharTok{\%\textgreater{}\%}
  \FunctionTok{group\_by}\NormalTok{(species, island) }\SpecialCharTok{\%\textgreater{}\%}
  \FunctionTok{na.omit}\NormalTok{() }\SpecialCharTok{\%\textgreater{}\%} \CommentTok{\#removes rows with NA values (a few rows may otherwise have NA due to sampling error in the field)}
  \FunctionTok{summarize}\NormalTok{(}\AttributeTok{meanbill=}\FunctionTok{mean}\NormalTok{(bill\_length\_mm), }\AttributeTok{sd=}\FunctionTok{sd}\NormalTok{(bill\_length\_mm), }\AttributeTok{n=}\FunctionTok{n}\NormalTok{(), }\AttributeTok{se=}\NormalTok{sd}\SpecialCharTok{/}\FunctionTok{sqrt}\NormalTok{(n))}

\NormalTok{sumpens}
\end{Highlighting}
\end{Shaded}

\begin{verbatim}
# A tibble: 5 x 6
# Groups:   species [3]
  species   island    meanbill    sd     n    se
  <fct>     <fct>        <dbl> <dbl> <int> <dbl>
1 Adelie    Biscoe        39.0  2.48    44 0.374
2 Adelie    Dream         38.5  2.48    55 0.335
3 Adelie    Torgersen     39.0  3.03    47 0.442
4 Chinstrap Dream         48.8  3.34    68 0.405
5 Gentoo    Biscoe        47.6  3.11   119 0.285
\end{verbatim}

\begin{Shaded}
\begin{Highlighting}[]
\CommentTok{\# Now we can plot! }
\FunctionTok{ggplot}\NormalTok{(}\AttributeTok{data=}\NormalTok{sumpens, }\FunctionTok{aes}\NormalTok{(}\AttributeTok{x=}\NormalTok{species, }\AttributeTok{y=}\NormalTok{meanbill, }\AttributeTok{color=}\NormalTok{island))}\SpecialCharTok{+}
  \FunctionTok{geom\_point}\NormalTok{()}\SpecialCharTok{+}
  \FunctionTok{geom\_errorbar}\NormalTok{(}\AttributeTok{data=}\NormalTok{sumpens, }\FunctionTok{aes}\NormalTok{(}\AttributeTok{x=}\NormalTok{species, }\AttributeTok{ymin=}\NormalTok{meanbill}\SpecialCharTok{{-}}\NormalTok{se, }\AttributeTok{ymax=}\NormalTok{meanbill}\SpecialCharTok{+}\NormalTok{se), }\AttributeTok{width=}\FloatTok{0.2}\NormalTok{)}
\end{Highlighting}
\end{Shaded}

\begin{figure}[H]

{\centering \includegraphics{Lab_2_files/figure-pdf/unnamed-chunk-18-1.pdf}

}

\end{figure}

And if we want to be extra fancy (and rigorous), we can plot the raw
data behind the mean+error This is considered a \textbf{graphical best
practice} as we can see the mean, error, and the true spread of the
data!

\begin{Shaded}
\begin{Highlighting}[]
\FunctionTok{ggplot}\NormalTok{()}\SpecialCharTok{+}
  \FunctionTok{geom\_jitter}\NormalTok{(}\AttributeTok{data=}\NormalTok{ penguins, }\FunctionTok{aes}\NormalTok{(}\AttributeTok{x=}\NormalTok{species, }\AttributeTok{y=}\NormalTok{bill\_length\_mm, }\AttributeTok{color=}\NormalTok{island), }\AttributeTok{alpha=}\FloatTok{0.5}\NormalTok{, }\AttributeTok{width=}\FloatTok{0.2}\NormalTok{)}\SpecialCharTok{+} \CommentTok{\#this is the raw data}
  \FunctionTok{geom\_point}\NormalTok{(}\AttributeTok{data=}\NormalTok{sumpens, }\FunctionTok{aes}\NormalTok{(}\AttributeTok{x=}\NormalTok{species, }\AttributeTok{y=}\NormalTok{meanbill, }\AttributeTok{color=}\NormalTok{island), }\AttributeTok{size=}\DecValTok{3}\NormalTok{)}\SpecialCharTok{+} \CommentTok{\#this is the averages}
  \FunctionTok{geom\_errorbar}\NormalTok{(}\AttributeTok{data=}\NormalTok{sumpens, }\FunctionTok{aes}\NormalTok{(}\AttributeTok{x=}\NormalTok{species, }\AttributeTok{ymin=}\NormalTok{meanbill}\SpecialCharTok{{-}}\NormalTok{se, }\AttributeTok{ymax=}\NormalTok{meanbill}\SpecialCharTok{+}\NormalTok{se), }\AttributeTok{width=}\FloatTok{0.1}\NormalTok{)}
\end{Highlighting}
\end{Shaded}

\begin{verbatim}
Warning: Removed 2 rows containing missing values or values outside the scale range
(`geom_point()`).
\end{verbatim}

\begin{figure}[H]

{\centering \includegraphics{Lab_2_files/figure-pdf/unnamed-chunk-19-1.pdf}

}

\end{figure}

An alternative to geom\_jitter, which doesn't always work, is to use
geom\_point but force the points to not overlap with position\_dodge.
Here is an example

\begin{Shaded}
\begin{Highlighting}[]
\CommentTok{\#first we should define the distance of our position\_dodge}
\NormalTok{pd}\OtherTok{\textless{}{-}}\FunctionTok{position\_dodge}\NormalTok{(}\AttributeTok{width=}\FloatTok{0.2}\NormalTok{)}

\FunctionTok{ggplot}\NormalTok{(}\AttributeTok{data=}\NormalTok{sumpens, }\FunctionTok{aes}\NormalTok{(}\AttributeTok{x=}\NormalTok{species, }\AttributeTok{y=}\NormalTok{meanbill, }\AttributeTok{color=}\NormalTok{island))}\SpecialCharTok{+}
  \FunctionTok{geom\_point}\NormalTok{(}\AttributeTok{data=}\NormalTok{ penguins, }\FunctionTok{aes}\NormalTok{(}\AttributeTok{x=}\NormalTok{species, }\AttributeTok{y=}\NormalTok{bill\_length\_mm, }\AttributeTok{color=}\NormalTok{island), }\AttributeTok{alpha=}\FloatTok{0.2}\NormalTok{, }\AttributeTok{width=}\FloatTok{0.2}\NormalTok{, }\AttributeTok{position=}\NormalTok{pd)}\SpecialCharTok{+} \CommentTok{\#raw data}
  \FunctionTok{geom\_point}\NormalTok{(}\AttributeTok{size=}\DecValTok{3}\NormalTok{, }\AttributeTok{position=}\NormalTok{pd)}\SpecialCharTok{+} \CommentTok{\#averages}
  \FunctionTok{geom\_errorbar}\NormalTok{(}\FunctionTok{aes}\NormalTok{(}\AttributeTok{ymin=}\NormalTok{meanbill}\SpecialCharTok{{-}}\NormalTok{se, }\AttributeTok{ymax=}\NormalTok{meanbill}\SpecialCharTok{+}\NormalTok{se), }\AttributeTok{width=}\FloatTok{0.2}\NormalTok{, }\AttributeTok{position=}\NormalTok{pd)}
\end{Highlighting}
\end{Shaded}

\begin{verbatim}
Warning in geom_point(data = penguins, aes(x = species, y = bill_length_mm, :
Ignoring unknown parameters: `width`
\end{verbatim}

\begin{verbatim}
Warning: Removed 2 rows containing missing values or values outside the scale range
(`geom_point()`).
\end{verbatim}

\begin{figure}[H]

{\centering \includegraphics{Lab_2_files/figure-pdf/unnamed-chunk-20-1.pdf}

}

\end{figure}

This code will produce the same graph as above. Note that in
geom\_jitter we just replaced width = with position =

\begin{Shaded}
\begin{Highlighting}[]
\FunctionTok{ggplot}\NormalTok{(sumpens, }\FunctionTok{aes}\NormalTok{(}\AttributeTok{x=}\NormalTok{species, }\AttributeTok{y=}\NormalTok{ meanbill, }\AttributeTok{color=}\NormalTok{island))}\SpecialCharTok{+}
  \FunctionTok{geom\_jitter}\NormalTok{(}\AttributeTok{data=}\NormalTok{ penguins, }\FunctionTok{aes}\NormalTok{(}\AttributeTok{x=}\NormalTok{species, }\AttributeTok{y=}\NormalTok{bill\_length\_mm, }\AttributeTok{color=}\NormalTok{island), }\AttributeTok{alpha=}\FloatTok{0.5}\NormalTok{, }\AttributeTok{position=}\NormalTok{pd)}\SpecialCharTok{+} \CommentTok{\#this is the raw data}
  \FunctionTok{geom\_point}\NormalTok{(}\AttributeTok{size=}\DecValTok{3}\NormalTok{,}\AttributeTok{position=}\NormalTok{pd)}\SpecialCharTok{+} \CommentTok{\#this is the averages}
  \FunctionTok{geom\_errorbar}\NormalTok{(}\FunctionTok{aes}\NormalTok{(}\AttributeTok{ymin=}\NormalTok{meanbill}\SpecialCharTok{{-}}\NormalTok{se, }\AttributeTok{ymax=}\NormalTok{meanbill}\SpecialCharTok{+}\NormalTok{se), }\AttributeTok{width=}\FloatTok{0.2}\NormalTok{, }\AttributeTok{position=}\NormalTok{pd)}
\end{Highlighting}
\end{Shaded}

\begin{verbatim}
Warning: Removed 2 rows containing missing values or values outside the scale range
(`geom_point()`).
\end{verbatim}

\begin{figure}[H]

{\centering \includegraphics{Lab_2_files/figure-pdf/unnamed-chunk-21-1.pdf}

}

\end{figure}

\hypertarget{intermediate-aesthetics}{%
\section{\texorpdfstring{\textbf{4.) Intermediate
Aesthetics}}{4.) Intermediate Aesthetics}}\label{intermediate-aesthetics}}

\subsection{\texorpdfstring{\textbf{titles and axis
labels}}{titles and axis labels}}

Titles and axis labels are easy to add and change in ggplot! We simply
add another line to our code. \textbf{NOTE} you can also add a subtitle,
caption, or change the legend title using labs!

\begin{Shaded}
\begin{Highlighting}[]
\FunctionTok{ggplot}\NormalTok{(}\AttributeTok{data=}\NormalTok{penguins, }\FunctionTok{aes}\NormalTok{(}\AttributeTok{x=}\NormalTok{species, }\AttributeTok{y=}\NormalTok{ bill\_length\_mm)) }\SpecialCharTok{+}
  \FunctionTok{geom\_boxplot}\NormalTok{(}\FunctionTok{aes}\NormalTok{(}\AttributeTok{fill=}\NormalTok{species))}\SpecialCharTok{+}
  \FunctionTok{scale\_fill\_aaas}\NormalTok{()}\SpecialCharTok{+}
  \FunctionTok{theme\_classic}\NormalTok{()}\SpecialCharTok{+}
  \FunctionTok{labs}\NormalTok{(}\AttributeTok{x =} \StringTok{\textquotesingle{}Species\textquotesingle{}}\NormalTok{, }\AttributeTok{y=}\StringTok{\textquotesingle{}Bill length (mm)\textquotesingle{}}\NormalTok{, }\AttributeTok{title=}\StringTok{\textquotesingle{}Penguin bill length by species\textquotesingle{}}\NormalTok{, }\AttributeTok{fill=}\StringTok{\textquotesingle{}Species\textquotesingle{}}\NormalTok{)}\SpecialCharTok{+} \CommentTok{\#here I change the x{-}axis and y{-}axis labels, add a title, and change the legend label (to capitalize the \textquotesingle{}S\textquotesingle{} in \textquotesingle{}species\textquotesingle{})}
  \FunctionTok{theme}\NormalTok{(}\AttributeTok{text=}\FunctionTok{element\_text}\NormalTok{(}\AttributeTok{size=}\DecValTok{18}\NormalTok{))}
\end{Highlighting}
\end{Shaded}

\begin{verbatim}
Warning: Removed 2 rows containing non-finite outside the scale range
(`stat_boxplot()`).
\end{verbatim}

\begin{figure}[H]

{\centering \includegraphics{Lab_2_files/figure-pdf/unnamed-chunk-22-1.pdf}

}

\end{figure}

\subsection{\texorpdfstring{\textbf{Colors}}{Colors}}

We can change colors conditionally or manually.

\textbf{Conditional Color Change} To change colors conditionally, we use
color= or fill= within an aes() call.

Here I have changed the outline color (color=) for a series of boxplots
based on species

\begin{Shaded}
\begin{Highlighting}[]
\FunctionTok{ggplot}\NormalTok{(}\AttributeTok{data=}\NormalTok{penguins, }\FunctionTok{aes}\NormalTok{(}\AttributeTok{x=}\NormalTok{species, }\AttributeTok{y=}\NormalTok{ bill\_length\_mm, }\AttributeTok{color=}\NormalTok{species)) }\SpecialCharTok{+}
  \FunctionTok{geom\_boxplot}\NormalTok{()}
\end{Highlighting}
\end{Shaded}

\begin{verbatim}
Warning: Removed 2 rows containing non-finite outside the scale range
(`stat_boxplot()`).
\end{verbatim}

\begin{figure}[H]

{\centering \includegraphics{Lab_2_files/figure-pdf/unnamed-chunk-23-1.pdf}

}

\end{figure}

I can also change the fill of the boxplots

\begin{Shaded}
\begin{Highlighting}[]
\FunctionTok{ggplot}\NormalTok{(}\AttributeTok{data=}\NormalTok{penguins, }\FunctionTok{aes}\NormalTok{(}\AttributeTok{x=}\NormalTok{species, }\AttributeTok{y=}\NormalTok{ bill\_length\_mm, }\AttributeTok{fill=}\NormalTok{species)) }\SpecialCharTok{+}
  \FunctionTok{geom\_boxplot}\NormalTok{()}
\end{Highlighting}
\end{Shaded}

\begin{verbatim}
Warning: Removed 2 rows containing non-finite outside the scale range
(`stat_boxplot()`).
\end{verbatim}

\begin{figure}[H]

{\centering \includegraphics{Lab_2_files/figure-pdf/unnamed-chunk-24-1.pdf}

}

\end{figure}

\textbf{Manual Color Change} We can also change colors manually by using
one of many options within ggplot. scale\_color\_manual (or
scale\_fill\_manual) is the easiest. We simply define colors we want to
use by name or hexcode.

\begin{Shaded}
\begin{Highlighting}[]
\FunctionTok{ggplot}\NormalTok{(}\AttributeTok{data=}\NormalTok{penguins, }\FunctionTok{aes}\NormalTok{(}\AttributeTok{x=}\NormalTok{species, }\AttributeTok{y=}\NormalTok{ bill\_length\_mm)) }\SpecialCharTok{+}
  \FunctionTok{geom\_boxplot}\NormalTok{(}\FunctionTok{aes}\NormalTok{(}\AttributeTok{fill=}\NormalTok{species))}\SpecialCharTok{+}
  \FunctionTok{scale\_fill\_manual}\NormalTok{(}\AttributeTok{values=}\FunctionTok{c}\NormalTok{(}\StringTok{\textquotesingle{}red\textquotesingle{}}\NormalTok{, }\StringTok{\textquotesingle{}black\textquotesingle{}}\NormalTok{, }\StringTok{\textquotesingle{}blue\textquotesingle{}}\NormalTok{))}
\end{Highlighting}
\end{Shaded}

\begin{verbatim}
Warning: Removed 2 rows containing non-finite outside the scale range
(`stat_boxplot()`).
\end{verbatim}

\begin{figure}[H]

{\centering \includegraphics{Lab_2_files/figure-pdf/unnamed-chunk-25-1.pdf}

}

\end{figure}

Here's a giant table of color options in ggplot
\includegraphics{images/ggplot_colors.png} You can also \textbf{make
your own color palette} and apply that to your figure!

\begin{Shaded}
\begin{Highlighting}[]
\NormalTok{mypal}\OtherTok{\textless{}{-}}\FunctionTok{c}\NormalTok{(}\StringTok{\textquotesingle{}dodgerblue\textquotesingle{}}\NormalTok{, }\StringTok{\textquotesingle{}forestgreen\textquotesingle{}}\NormalTok{, }\StringTok{\textquotesingle{}coral\textquotesingle{}}\NormalTok{) }\CommentTok{\# here I\textquotesingle{}ve made a 3 color palette}

\FunctionTok{ggplot}\NormalTok{(}\AttributeTok{data=}\NormalTok{penguins, }\FunctionTok{aes}\NormalTok{(}\AttributeTok{x=}\NormalTok{species, }\AttributeTok{y=}\NormalTok{ bill\_length\_mm)) }\SpecialCharTok{+}
  \FunctionTok{geom\_boxplot}\NormalTok{(}\FunctionTok{aes}\NormalTok{(}\AttributeTok{fill=}\NormalTok{species))}\SpecialCharTok{+}
  \FunctionTok{scale\_fill\_manual}\NormalTok{(}\AttributeTok{values=}\NormalTok{mypal)}
\end{Highlighting}
\end{Shaded}

\begin{verbatim}
Warning: Removed 2 rows containing non-finite outside the scale range
(`stat_boxplot()`).
\end{verbatim}

\begin{figure}[H]

{\centering \includegraphics{Lab_2_files/figure-pdf/unnamed-chunk-26-1.pdf}

}

\end{figure}

You can use the package
\href{https://renenyffenegger.ch/notes/development/languages/R/packages/RColorBrewer/index}{RColorBrewer}
to make palettes as well. I'll let you explore that one on your own!

Finally, EASY and nice looking palettes with
\href{https://cran.r-project.org/web/packages/ggsci/vignettes/ggsci.html}{ggsci}
ggsci is a simple and neat package that allows us to use scientific
journal color themes for our data (usually colorblind friendly and nice
looking). we simply change our ``scale\_color\_manual'' to
``scale\_color\_palname'' where ``palname'' is one of many provided by
ggsci. For example, we might use scale\_color\_aaas()

\begin{Shaded}
\begin{Highlighting}[]
\FunctionTok{ggplot}\NormalTok{(}\AttributeTok{data=}\NormalTok{penguins, }\FunctionTok{aes}\NormalTok{(}\AttributeTok{x=}\NormalTok{species, }\AttributeTok{y=}\NormalTok{ bill\_length\_mm)) }\SpecialCharTok{+}
  \FunctionTok{geom\_boxplot}\NormalTok{(}\FunctionTok{aes}\NormalTok{(}\AttributeTok{fill=}\NormalTok{species))}\SpecialCharTok{+}
  \FunctionTok{scale\_fill\_aaas}\NormalTok{()}
\end{Highlighting}
\end{Shaded}

\begin{verbatim}
Warning: Removed 2 rows containing non-finite outside the scale range
(`stat_boxplot()`).
\end{verbatim}

\begin{figure}[H]

{\centering \includegraphics{Lab_2_files/figure-pdf/unnamed-chunk-27-1.pdf}

}

\end{figure}

\subsection{\texorpdfstring{\textbf{Shapes}}{Shapes}}

ggplot gives us options to change point shape using the aesethic option
`shape' We can either change shape based on a characterstic of the data
(`cyl', for example), make all the shapes the same, or manually control
shape

Below is a table of shape options:

\begin{figure}

{\centering \includegraphics{C:/Users/Justin Baumann/Desktop/r_for_bioeco/intro_r_for_bio_eco/images/ggplot_shapes.png}

}

\caption{ggplot shape options}

\end{figure}

\textbf{Conditional Shape Change}

\begin{Shaded}
\begin{Highlighting}[]
\FunctionTok{ggplot}\NormalTok{(}\AttributeTok{data=}\NormalTok{penguins, }\FunctionTok{aes}\NormalTok{(}\AttributeTok{x=}\NormalTok{species, }\AttributeTok{y=}\NormalTok{bill\_length\_mm, }\AttributeTok{color=}\NormalTok{island, }\AttributeTok{shape=}\NormalTok{island))}\SpecialCharTok{+} 
  \FunctionTok{geom\_jitter}\NormalTok{(}\AttributeTok{size=}\DecValTok{2}\NormalTok{)}\SpecialCharTok{+}
  \FunctionTok{theme\_classic}\NormalTok{()}
\end{Highlighting}
\end{Shaded}

\begin{verbatim}
Warning: Removed 2 rows containing missing values or values outside the scale range
(`geom_point()`).
\end{verbatim}

\begin{figure}[H]

{\centering \includegraphics{Lab_2_files/figure-pdf/unnamed-chunk-28-1.pdf}

}

\end{figure}

\textbf{Change all shapes to triangles}

\begin{Shaded}
\begin{Highlighting}[]
\FunctionTok{ggplot}\NormalTok{(}\AttributeTok{data=}\NormalTok{mtcars, }\FunctionTok{aes}\NormalTok{(}\AttributeTok{x=}\NormalTok{cyl, }\AttributeTok{y=}\NormalTok{mpg, }\AttributeTok{color=}\NormalTok{cyl))}\SpecialCharTok{+}
  \FunctionTok{geom\_point}\NormalTok{(}\AttributeTok{shape=}\DecValTok{17}\NormalTok{) }\CommentTok{\#Here \textquotesingle{}shape=\textquotesingle{} is inside the settings for geom\_point. Note that it is outside the aes() function, as that applied aesethics conditionally)}
\end{Highlighting}
\end{Shaded}

\begin{figure}[H]

{\centering \includegraphics{Lab_2_files/figure-pdf/unnamed-chunk-29-1.pdf}

}

\end{figure}

\begin{Shaded}
\begin{Highlighting}[]
\CommentTok{\#example 2, same w/ different syntax}
\FunctionTok{ggplot}\NormalTok{()}\SpecialCharTok{+}
  \FunctionTok{geom\_point}\NormalTok{(}\AttributeTok{data=}\NormalTok{mtcars, }\FunctionTok{aes}\NormalTok{(}\AttributeTok{x=}\NormalTok{cyl, }\AttributeTok{y=}\NormalTok{mpg, }\AttributeTok{color=}\NormalTok{cyl), }\AttributeTok{shape=}\DecValTok{17}\NormalTok{)}
\end{Highlighting}
\end{Shaded}

\begin{figure}[H]

{\centering \includegraphics{Lab_2_files/figure-pdf/unnamed-chunk-29-2.pdf}

}

\end{figure}

\textbf{Manual shape changes}

\begin{Shaded}
\begin{Highlighting}[]
\FunctionTok{ggplot}\NormalTok{(}\AttributeTok{data=}\NormalTok{penguins, }\FunctionTok{aes}\NormalTok{(}\AttributeTok{x=}\NormalTok{species, }\AttributeTok{y=}\NormalTok{bill\_length\_mm, }\AttributeTok{color=}\NormalTok{island, }\AttributeTok{shape=}\NormalTok{island))}\SpecialCharTok{+} 
  \FunctionTok{geom\_jitter}\NormalTok{(}\AttributeTok{size=}\DecValTok{2}\NormalTok{)}\SpecialCharTok{+}
  \FunctionTok{theme\_classic}\NormalTok{()}\SpecialCharTok{+}  
  \FunctionTok{scale\_shape\_manual}\NormalTok{(}\AttributeTok{values=}\FunctionTok{c}\NormalTok{(}\DecValTok{2}\NormalTok{,}\DecValTok{3}\NormalTok{,}\DecValTok{4}\NormalTok{)) }\CommentTok{\#scale\_shape\_manual allows us to choose shapes for each group (cyl in this case). c stands for concatenate, as we\textquotesingle{}ve seen before}
\end{Highlighting}
\end{Shaded}

\begin{verbatim}
Warning: Removed 2 rows containing missing values or values outside the scale range
(`geom_point()`).
\end{verbatim}

\begin{figure}[H]

{\centering \includegraphics{Lab_2_files/figure-pdf/unnamed-chunk-30-1.pdf}

}

\end{figure}

\textbf{Changing Size of points}

Conditional Shape Change

\begin{Shaded}
\begin{Highlighting}[]
\FunctionTok{ggplot}\NormalTok{(}\AttributeTok{data=}\NormalTok{mtcars, }\FunctionTok{aes}\NormalTok{(}\AttributeTok{x=}\NormalTok{cyl, }\AttributeTok{y=}\NormalTok{mpg, }\AttributeTok{color=}\NormalTok{cyl, }\AttributeTok{size=}\NormalTok{cyl))}\SpecialCharTok{+} \CommentTok{\#note that we added \textquotesingle{}size=\textquotesingle{} to our aes. }
  \FunctionTok{geom\_point}\NormalTok{()}
\end{Highlighting}
\end{Shaded}

\begin{figure}[H]

{\centering \includegraphics{Lab_2_files/figure-pdf/unnamed-chunk-31-1.pdf}

}

\end{figure}

\begin{Shaded}
\begin{Highlighting}[]
\CommentTok{\#note the warning message that using size for a discrete variable is not best practice. }
\CommentTok{\#Instead, let\textquotesingle{}s use the size to five us an idea of hp (a 3rd variable)}

\FunctionTok{ggplot}\NormalTok{(}\AttributeTok{data=}\NormalTok{mtcars, }\FunctionTok{aes}\NormalTok{(}\AttributeTok{x=}\NormalTok{cyl, }\AttributeTok{y=}\NormalTok{mpg, }\AttributeTok{color=}\NormalTok{cyl, }\AttributeTok{size=}\NormalTok{hp))}\SpecialCharTok{+} \CommentTok{\#note that we added \textquotesingle{}size=\textquotesingle{} to our aes. }
  \FunctionTok{geom\_point}\NormalTok{()}
\end{Highlighting}
\end{Shaded}

\begin{figure}[H]

{\centering \includegraphics{Lab_2_files/figure-pdf/unnamed-chunk-31-2.pdf}

}

\end{figure}

Change size of all points (all points must be same size)

\begin{Shaded}
\begin{Highlighting}[]
\FunctionTok{ggplot}\NormalTok{(}\AttributeTok{data=}\NormalTok{mtcars, }\FunctionTok{aes}\NormalTok{(}\AttributeTok{x=}\NormalTok{cyl, }\AttributeTok{y=}\NormalTok{mpg, }\AttributeTok{color=}\NormalTok{cyl))}\SpecialCharTok{+}  
  \FunctionTok{geom\_point}\NormalTok{(}\AttributeTok{size=}\DecValTok{5}\NormalTok{) }\CommentTok{\#as w/ shape, point needs to be outside the aes() here. }
\end{Highlighting}
\end{Shaded}

\begin{figure}[H]

{\centering \includegraphics{Lab_2_files/figure-pdf/unnamed-chunk-32-1.pdf}

}

\end{figure}

\subsection{\texorpdfstring{\textbf{Facets}}{Facets}}

Often in science we are interested in comparing several graphs at once
or looking at 3 or 4 variables at a time. This means we may want to have
multi-panel graphs or multiple graphs on the same page. While it is
common to produce graphs in R and combine them into ``final'' manuscript
ready version in other programs, such as Adobe Illustrator or Inkscape
(a free alternative to Illustrator), producing manuscript quality
figures in R is possible! In fact, it is only getting easier, thanks to
some new packages (like patchwork). Below I will show you how to make
multipanel figures (aka facets) and how to put many figures on one page
(using the patchwork package-- the easiest of the many options for doing
this).\\
Facets allow us to produce multiple graph panels with one ggplot code.
We can separate out a variable for easier viewing or even create a grid
of graphs using multiple variables.

facet\_wrap() allows us to make multiple panels. The panels are aligned
in columns and rows. We need to use `\textasciitilde{}' in our
facet\_wrap code. The `\textasciitilde{}' essentially means ``by''

\begin{Shaded}
\begin{Highlighting}[]
\FunctionTok{ggplot}\NormalTok{(}\AttributeTok{data=}\NormalTok{penguins, }\FunctionTok{aes}\NormalTok{(}\AttributeTok{x=}\NormalTok{island, }\AttributeTok{y=}\NormalTok{ bill\_length\_mm, }\AttributeTok{fill=}\NormalTok{species)) }\SpecialCharTok{+}
  \FunctionTok{geom\_boxplot}\NormalTok{()}\SpecialCharTok{+}
  \FunctionTok{facet\_wrap}\NormalTok{(}\SpecialCharTok{\textasciitilde{}}\NormalTok{island)}\SpecialCharTok{+}
  \FunctionTok{scale\_color\_aaas}\NormalTok{()}\SpecialCharTok{+}
  \FunctionTok{theme\_classic}\NormalTok{()}
\end{Highlighting}
\end{Shaded}

\begin{verbatim}
Warning: Removed 2 rows containing non-finite outside the scale range
(`stat_boxplot()`).
\end{verbatim}

\begin{figure}[H]

{\centering \includegraphics{Lab_2_files/figure-pdf/unnamed-chunk-33-1.pdf}

}

\end{figure}

We can specify the number of columns and rows we want to built the
panels how we like them

\begin{Shaded}
\begin{Highlighting}[]
\FunctionTok{ggplot}\NormalTok{(}\AttributeTok{data=}\NormalTok{penguins, }\FunctionTok{aes}\NormalTok{(}\AttributeTok{x=}\NormalTok{year, }\AttributeTok{y=}\NormalTok{ bill\_length\_mm, }\AttributeTok{fill=}\NormalTok{species)) }\SpecialCharTok{+}
  \FunctionTok{geom\_boxplot}\NormalTok{()}\SpecialCharTok{+}
  \FunctionTok{facet\_wrap}\NormalTok{(}\SpecialCharTok{\textasciitilde{}}\NormalTok{island, }\AttributeTok{ncol=}\DecValTok{2}\NormalTok{)}\SpecialCharTok{+} \CommentTok{\#2 columns }
  \FunctionTok{scale\_color\_aaas}\NormalTok{()}\SpecialCharTok{+}
  \FunctionTok{theme\_classic}\NormalTok{()}
\end{Highlighting}
\end{Shaded}

\begin{verbatim}
Warning: Removed 2 rows containing non-finite outside the scale range
(`stat_boxplot()`).
\end{verbatim}

\begin{figure}[H]

{\centering \includegraphics{Lab_2_files/figure-pdf/unnamed-chunk-34-1.pdf}

}

\end{figure}

\begin{Shaded}
\begin{Highlighting}[]
\FunctionTok{ggplot}\NormalTok{(}\AttributeTok{data=}\NormalTok{penguins, }\FunctionTok{aes}\NormalTok{(}\AttributeTok{x=}\NormalTok{year, }\AttributeTok{y=}\NormalTok{ bill\_length\_mm, }\AttributeTok{fill=}\NormalTok{species)) }\SpecialCharTok{+}
  \FunctionTok{geom\_boxplot}\NormalTok{()}\SpecialCharTok{+}
  \FunctionTok{facet\_wrap}\NormalTok{(}\SpecialCharTok{\textasciitilde{}}\NormalTok{island, }\AttributeTok{nrow=}\DecValTok{3}\NormalTok{)}\SpecialCharTok{+} \CommentTok{\#3 rows}
  \FunctionTok{scale\_color\_aaas}\NormalTok{()}\SpecialCharTok{+}
  \FunctionTok{theme\_classic}\NormalTok{()}
\end{Highlighting}
\end{Shaded}

\begin{verbatim}
Warning: Removed 2 rows containing non-finite outside the scale range
(`stat_boxplot()`).
\end{verbatim}

\begin{figure}[H]

{\centering \includegraphics{Lab_2_files/figure-pdf/unnamed-chunk-35-1.pdf}

}

\end{figure}

We can even use a formula for building our facets if we'd like!

\begin{Shaded}
\begin{Highlighting}[]
\FunctionTok{ggplot}\NormalTok{(}\AttributeTok{data=}\NormalTok{penguins, }\FunctionTok{aes}\NormalTok{(}\AttributeTok{x=}\NormalTok{island, }\AttributeTok{y=}\NormalTok{ bill\_length\_mm, }\AttributeTok{fill=}\NormalTok{species)) }\SpecialCharTok{+}
  \FunctionTok{geom\_boxplot}\NormalTok{()}\SpecialCharTok{+}
  \FunctionTok{facet\_wrap}\NormalTok{(}\SpecialCharTok{\textasciitilde{}}\NormalTok{species}\SpecialCharTok{+}\NormalTok{year)}\SpecialCharTok{+}
  \FunctionTok{scale\_color\_aaas}\NormalTok{()}\SpecialCharTok{+}
  \FunctionTok{theme\_classic}\NormalTok{()}
\end{Highlighting}
\end{Shaded}

\begin{verbatim}
Warning: Removed 2 rows containing non-finite outside the scale range
(`stat_boxplot()`).
\end{verbatim}

\begin{figure}[H]

{\centering \includegraphics{Lab_2_files/figure-pdf/unnamed-chunk-36-1.pdf}

}

\end{figure}

\subsection{\texorpdfstring{\textbf{Multiple plots on the same
page}}{Multiple plots on the same page}}

Using the simple and wonderful patchwork package, we can place multiple
plots on the same page. To do this, we must actually name each plot.
Here's an example.

Patchwork is super easy! Learn more
\href{https://patchwork.data-imaginist.com/articles/patchwork.html}{here}(with
examples)

First, let's make some graphs and name them

\begin{Shaded}
\begin{Highlighting}[]
\CommentTok{\#First, we need to calculate a mean bill length for our penguins by species and island}
\NormalTok{sumpens}\OtherTok{\textless{}{-}}\NormalTok{ penguins }\SpecialCharTok{\%\textgreater{}\%}
  \FunctionTok{group\_by}\NormalTok{(species, island) }\SpecialCharTok{\%\textgreater{}\%}
  \FunctionTok{na.omit}\NormalTok{() }\SpecialCharTok{\%\textgreater{}\%} \CommentTok{\#removes rows with NA values (a few rows may otherwise have NA due to sampling error in the field)}
  \FunctionTok{summarize}\NormalTok{(}\AttributeTok{meanbill=}\FunctionTok{mean}\NormalTok{(bill\_length\_mm), }\AttributeTok{sd=}\FunctionTok{sd}\NormalTok{(bill\_length\_mm), }\AttributeTok{n=}\FunctionTok{n}\NormalTok{(), }\AttributeTok{se=}\NormalTok{sd}\SpecialCharTok{/}\FunctionTok{sqrt}\NormalTok{(n))}
\end{Highlighting}
\end{Shaded}

\begin{verbatim}
`summarise()` has grouped output by 'species'. You can override using the
`.groups` argument.
\end{verbatim}

\begin{Shaded}
\begin{Highlighting}[]
\NormalTok{sumpens}
\end{Highlighting}
\end{Shaded}

\begin{verbatim}
# A tibble: 5 x 6
# Groups:   species [3]
  species   island    meanbill    sd     n    se
  <fct>     <fct>        <dbl> <dbl> <int> <dbl>
1 Adelie    Biscoe        39.0  2.48    44 0.374
2 Adelie    Dream         38.5  2.48    55 0.335
3 Adelie    Torgersen     39.0  3.03    47 0.442
4 Chinstrap Dream         48.8  3.34    68 0.405
5 Gentoo    Biscoe        47.6  3.11   119 0.285
\end{verbatim}

\begin{Shaded}
\begin{Highlighting}[]
\CommentTok{\# Next, we can make our graphs!}

\NormalTok{p1}\OtherTok{\textless{}{-}}\FunctionTok{ggplot}\NormalTok{(}\AttributeTok{data=}\NormalTok{penguins, }\FunctionTok{aes}\NormalTok{(bill\_length\_mm))}\SpecialCharTok{+}
  \FunctionTok{geom\_histogram}\NormalTok{()}\SpecialCharTok{+}
  \FunctionTok{theme\_classic}\NormalTok{()}


\NormalTok{p2}\OtherTok{\textless{}{-}}\FunctionTok{ggplot}\NormalTok{()}\SpecialCharTok{+}
  \FunctionTok{geom\_jitter}\NormalTok{(}\AttributeTok{data=}\NormalTok{ penguins, }\FunctionTok{aes}\NormalTok{(}\AttributeTok{x=}\NormalTok{species, }\AttributeTok{y=}\NormalTok{bill\_length\_mm, }\AttributeTok{color=}\NormalTok{island), }\AttributeTok{alpha=}\FloatTok{0.5}\NormalTok{, }\AttributeTok{width=}\FloatTok{0.2}\NormalTok{)}\SpecialCharTok{+}
  \FunctionTok{geom\_point}\NormalTok{(}\AttributeTok{data=}\NormalTok{sumpens, }\FunctionTok{aes}\NormalTok{(}\AttributeTok{x=}\NormalTok{species, }\AttributeTok{y=}\NormalTok{meanbill, }\AttributeTok{color=}\NormalTok{island), }\AttributeTok{size=}\DecValTok{3}\NormalTok{)}\SpecialCharTok{+}
  \FunctionTok{geom\_errorbar}\NormalTok{(}\AttributeTok{data=}\NormalTok{sumpens, }\FunctionTok{aes}\NormalTok{(}\AttributeTok{x=}\NormalTok{species, }\AttributeTok{ymin=}\NormalTok{meanbill}\SpecialCharTok{{-}}\NormalTok{se, }\AttributeTok{ymax=}\NormalTok{meanbill}\SpecialCharTok{+}\NormalTok{se), }\AttributeTok{width=}\FloatTok{0.1}\NormalTok{)}\SpecialCharTok{+}
  \FunctionTok{theme\_classic}\NormalTok{()}\SpecialCharTok{+}
  \FunctionTok{scale\_color\_aaas}\NormalTok{()}

\NormalTok{p3}\OtherTok{\textless{}{-}}\FunctionTok{ggplot}\NormalTok{(}\AttributeTok{data=}\NormalTok{penguins, }\FunctionTok{aes}\NormalTok{(island)) }\SpecialCharTok{+}
  \FunctionTok{geom\_bar}\NormalTok{(}\FunctionTok{aes}\NormalTok{(}\AttributeTok{fill=}\NormalTok{species), }\AttributeTok{position=} \FunctionTok{position\_dodge}\NormalTok{())}\SpecialCharTok{+}
  \FunctionTok{theme\_classic}\NormalTok{()}\SpecialCharTok{+}
  \FunctionTok{scale\_fill\_aaas}\NormalTok{()}
\end{Highlighting}
\end{Shaded}

Now let's patchwork them together! We make a simple formula to make a
patchwork. Addition puts everything in the same row. But we can use
division and other symbols to organize.

\begin{Shaded}
\begin{Highlighting}[]
\FunctionTok{library}\NormalTok{(patchwork)}

\NormalTok{p1}\SpecialCharTok{+}\NormalTok{p2}\SpecialCharTok{+}\NormalTok{p3}
\end{Highlighting}
\end{Shaded}

\begin{verbatim}
`stat_bin()` using `bins = 30`. Pick better value with `binwidth`.
\end{verbatim}

\begin{verbatim}
Warning: Removed 2 rows containing non-finite outside the scale range
(`stat_bin()`).
\end{verbatim}

\begin{verbatim}
Warning: Removed 2 rows containing missing values or values outside the scale range
(`geom_point()`).
\end{verbatim}

\begin{figure}[H]

{\centering \includegraphics{Lab_2_files/figure-pdf/unnamed-chunk-38-1.pdf}

}

\end{figure}

Division allows us to put panels in columns

\begin{Shaded}
\begin{Highlighting}[]
\NormalTok{p1}\SpecialCharTok{/}\NormalTok{p2}\SpecialCharTok{/}\NormalTok{p3}
\end{Highlighting}
\end{Shaded}

\begin{verbatim}
`stat_bin()` using `bins = 30`. Pick better value with `binwidth`.
\end{verbatim}

\begin{verbatim}
Warning: Removed 2 rows containing non-finite outside the scale range
(`stat_bin()`).
\end{verbatim}

\begin{verbatim}
Warning: Removed 2 rows containing missing values or values outside the scale range
(`geom_point()`).
\end{verbatim}

\begin{figure}[H]

{\centering \includegraphics{Lab_2_files/figure-pdf/unnamed-chunk-39-1.pdf}

}

\end{figure}

We can also combine addition and division (order of operations is still
a thing!)

\begin{Shaded}
\begin{Highlighting}[]
\NormalTok{(p1}\SpecialCharTok{+}\NormalTok{p2) }\SpecialCharTok{/}\NormalTok{ p3}
\end{Highlighting}
\end{Shaded}

\begin{verbatim}
`stat_bin()` using `bins = 30`. Pick better value with `binwidth`.
\end{verbatim}

\begin{verbatim}
Warning: Removed 2 rows containing non-finite outside the scale range
(`stat_bin()`).
\end{verbatim}

\begin{verbatim}
Warning: Removed 2 rows containing missing values or values outside the scale range
(`geom_point()`).
\end{verbatim}

\begin{figure}[H]

{\centering \includegraphics{Lab_2_files/figure-pdf/unnamed-chunk-40-1.pdf}

}

\end{figure}

There are other functions in patchwork that allow us to annotate plots,
give them labels, move/combine legends, etc.

\subsection{\texorpdfstring{\textbf{Themes}}{Themes}}

Themes allow us to change the background color and most other aspects of
a plot. There are a range of theme options within ggplot that will allow
us to quickly make clean plots. The two that are most commonly used are
theme\_bw() and theme\_classic()

\textbf{Default theme} (with terrible gray background)

\begin{Shaded}
\begin{Highlighting}[]
\FunctionTok{ggplot}\NormalTok{(}\AttributeTok{data=}\NormalTok{penguins, }\FunctionTok{aes}\NormalTok{(}\AttributeTok{x=}\NormalTok{species, }\AttributeTok{y=}\NormalTok{ bill\_length\_mm)) }\SpecialCharTok{+}
  \FunctionTok{geom\_boxplot}\NormalTok{(}\FunctionTok{aes}\NormalTok{(}\AttributeTok{fill=}\NormalTok{species))}\SpecialCharTok{+}
  \FunctionTok{scale\_fill\_aaas}\NormalTok{()}\SpecialCharTok{+}
  \FunctionTok{labs}\NormalTok{(}\AttributeTok{x =} \StringTok{\textquotesingle{}Species\textquotesingle{}}\NormalTok{, }\AttributeTok{y=}\StringTok{\textquotesingle{}Bill length (mm)\textquotesingle{}}\NormalTok{, }\AttributeTok{title=}\StringTok{\textquotesingle{}Penguin bill length by species\textquotesingle{}}\NormalTok{)}
\end{Highlighting}
\end{Shaded}

\begin{verbatim}
Warning: Removed 2 rows containing non-finite outside the scale range
(`stat_boxplot()`).
\end{verbatim}

\begin{figure}[H]

{\centering \includegraphics{Lab_2_files/figure-pdf/unnamed-chunk-41-1.pdf}

}

\end{figure}

\textbf{theme\_bw()} (removes gray background)

\begin{Shaded}
\begin{Highlighting}[]
\FunctionTok{ggplot}\NormalTok{(}\AttributeTok{data=}\NormalTok{penguins, }\FunctionTok{aes}\NormalTok{(}\AttributeTok{x=}\NormalTok{species, }\AttributeTok{y=}\NormalTok{ bill\_length\_mm)) }\SpecialCharTok{+}
  \FunctionTok{geom\_boxplot}\NormalTok{(}\FunctionTok{aes}\NormalTok{(}\AttributeTok{fill=}\NormalTok{species))}\SpecialCharTok{+}
  \FunctionTok{scale\_fill\_aaas}\NormalTok{()}\SpecialCharTok{+}
  \FunctionTok{labs}\NormalTok{(}\AttributeTok{x =} \StringTok{\textquotesingle{}Species\textquotesingle{}}\NormalTok{, }\AttributeTok{y=}\StringTok{\textquotesingle{}Bill length (mm)\textquotesingle{}}\NormalTok{, }\AttributeTok{title=}\StringTok{\textquotesingle{}Penguin bill length by species\textquotesingle{}}\NormalTok{)}\SpecialCharTok{+}
  \FunctionTok{theme\_bw}\NormalTok{()}
\end{Highlighting}
\end{Shaded}

\begin{verbatim}
Warning: Removed 2 rows containing non-finite outside the scale range
(`stat_boxplot()`).
\end{verbatim}

\begin{figure}[H]

{\centering \includegraphics{Lab_2_files/figure-pdf/unnamed-chunk-42-1.pdf}

}

\end{figure}

\textbf{theme\_classic()} (removes gray and grid lines)

\begin{Shaded}
\begin{Highlighting}[]
\FunctionTok{ggplot}\NormalTok{(}\AttributeTok{data=}\NormalTok{penguins, }\FunctionTok{aes}\NormalTok{(}\AttributeTok{x=}\NormalTok{species, }\AttributeTok{y=}\NormalTok{ bill\_length\_mm)) }\SpecialCharTok{+}
  \FunctionTok{geom\_boxplot}\NormalTok{(}\FunctionTok{aes}\NormalTok{(}\AttributeTok{fill=}\NormalTok{species))}\SpecialCharTok{+}
  \FunctionTok{scale\_fill\_aaas}\NormalTok{()}\SpecialCharTok{+}
  \FunctionTok{labs}\NormalTok{(}\AttributeTok{x =} \StringTok{\textquotesingle{}Species\textquotesingle{}}\NormalTok{, }\AttributeTok{y=}\StringTok{\textquotesingle{}Bill length (mm)\textquotesingle{}}\NormalTok{, }\AttributeTok{title=}\StringTok{\textquotesingle{}Penguin bill length by species\textquotesingle{}}\NormalTok{)}\SpecialCharTok{+}
  \FunctionTok{theme\_classic}\NormalTok{()}
\end{Highlighting}
\end{Shaded}

\begin{verbatim}
Warning: Removed 2 rows containing non-finite outside the scale range
(`stat_boxplot()`).
\end{verbatim}

\begin{figure}[H]

{\centering \includegraphics{Lab_2_files/figure-pdf/unnamed-chunk-43-1.pdf}

}

\end{figure}

The theme() function in ggplot is SUPER flexible. You can pretty much do
anything with it. This is key for customizing plots. I'd encourage you
to play around with this a bit.
\href{https://ggplot2.tidyverse.org/reference/theme.html}{Here} is a
great place to learn more and see examples.\\
\#\#\textbf{Some examples of using theme()}

\textbf{Changing text size}

\begin{Shaded}
\begin{Highlighting}[]
\FunctionTok{ggplot}\NormalTok{(}\AttributeTok{data=}\NormalTok{penguins, }\FunctionTok{aes}\NormalTok{(}\AttributeTok{x=}\NormalTok{species, }\AttributeTok{y=}\NormalTok{ bill\_length\_mm)) }\SpecialCharTok{+}
  \FunctionTok{geom\_boxplot}\NormalTok{(}\FunctionTok{aes}\NormalTok{(}\AttributeTok{fill=}\NormalTok{species))}\SpecialCharTok{+}
  \FunctionTok{scale\_fill\_aaas}\NormalTok{()}\SpecialCharTok{+}
  \FunctionTok{labs}\NormalTok{(}\AttributeTok{x =} \StringTok{\textquotesingle{}Species\textquotesingle{}}\NormalTok{, }\AttributeTok{y=}\StringTok{\textquotesingle{}Bill length (mm)\textquotesingle{}}\NormalTok{, }\AttributeTok{title=}\StringTok{\textquotesingle{}Penguin bill length by species\textquotesingle{}}\NormalTok{)}\SpecialCharTok{+}
  \FunctionTok{theme}\NormalTok{(}\AttributeTok{text=}\FunctionTok{element\_text}\NormalTok{(}\AttributeTok{size=}\DecValTok{24}\NormalTok{))}
\end{Highlighting}
\end{Shaded}

\begin{verbatim}
Warning: Removed 2 rows containing non-finite outside the scale range
(`stat_boxplot()`).
\end{verbatim}

\begin{figure}[H]

{\centering \includegraphics{Lab_2_files/figure-pdf/unnamed-chunk-44-1.pdf}

}

\end{figure}

\textbf{Remove the gray background}

\begin{Shaded}
\begin{Highlighting}[]
\FunctionTok{ggplot}\NormalTok{(}\AttributeTok{data=}\NormalTok{penguins, }\FunctionTok{aes}\NormalTok{(}\AttributeTok{x=}\NormalTok{species, }\AttributeTok{y=}\NormalTok{ bill\_length\_mm)) }\SpecialCharTok{+}
  \FunctionTok{geom\_boxplot}\NormalTok{(}\FunctionTok{aes}\NormalTok{(}\AttributeTok{fill=}\NormalTok{species))}\SpecialCharTok{+}
  \FunctionTok{scale\_fill\_aaas}\NormalTok{()}\SpecialCharTok{+}
  \FunctionTok{labs}\NormalTok{(}\AttributeTok{x =} \StringTok{\textquotesingle{}Species\textquotesingle{}}\NormalTok{, }\AttributeTok{y=}\StringTok{\textquotesingle{}Bill length (mm)\textquotesingle{}}\NormalTok{, }\AttributeTok{title=}\StringTok{\textquotesingle{}Penguin bill length by species\textquotesingle{}}\NormalTok{)}\SpecialCharTok{+}
  \FunctionTok{theme}\NormalTok{(}\AttributeTok{text=}\FunctionTok{element\_text}\NormalTok{(}\AttributeTok{size=}\DecValTok{24}\NormalTok{), }\AttributeTok{panel.background =} \FunctionTok{element\_rect}\NormalTok{(}\AttributeTok{fill=}\StringTok{"white"}\NormalTok{)) }\CommentTok{\#can use any color}
\end{Highlighting}
\end{Shaded}

\begin{verbatim}
Warning: Removed 2 rows containing non-finite outside the scale range
(`stat_boxplot()`).
\end{verbatim}

\begin{figure}[H]

{\centering \includegraphics{Lab_2_files/figure-pdf/unnamed-chunk-45-1.pdf}

}

\end{figure}

\textbf{Turn the X-Axis text}

\begin{Shaded}
\begin{Highlighting}[]
\FunctionTok{ggplot}\NormalTok{(}\AttributeTok{data=}\NormalTok{penguins, }\FunctionTok{aes}\NormalTok{(}\AttributeTok{x=}\NormalTok{species, }\AttributeTok{y=}\NormalTok{ bill\_length\_mm)) }\SpecialCharTok{+}
  \FunctionTok{geom\_boxplot}\NormalTok{(}\FunctionTok{aes}\NormalTok{(}\AttributeTok{fill=}\NormalTok{species))}\SpecialCharTok{+}
  \FunctionTok{scale\_fill\_aaas}\NormalTok{()}\SpecialCharTok{+}
  \FunctionTok{labs}\NormalTok{(}\AttributeTok{x =} \StringTok{\textquotesingle{}Species\textquotesingle{}}\NormalTok{, }\AttributeTok{y=}\StringTok{\textquotesingle{}Bill length (mm)\textquotesingle{}}\NormalTok{, }\AttributeTok{title=}\StringTok{\textquotesingle{}Penguin bill length by species\textquotesingle{}}\NormalTok{)}\SpecialCharTok{+}
  \FunctionTok{theme}\NormalTok{(}\AttributeTok{text=}\FunctionTok{element\_text}\NormalTok{(}\AttributeTok{size=}\DecValTok{24}\NormalTok{), }\AttributeTok{panel.background =} \FunctionTok{element\_rect}\NormalTok{(}\AttributeTok{fill=}\StringTok{"white"}\NormalTok{), }\AttributeTok{axis.text.x=}\FunctionTok{element\_text}\NormalTok{(}\AttributeTok{angle=}\DecValTok{90}\NormalTok{, }\AttributeTok{vjust=}\FloatTok{0.5}\NormalTok{, }\AttributeTok{hjust=}\FloatTok{0.8}\NormalTok{)) }\CommentTok{\#can adjust vertical and horizontal text positions}
\end{Highlighting}
\end{Shaded}

\begin{verbatim}
Warning: Removed 2 rows containing non-finite outside the scale range
(`stat_boxplot()`).
\end{verbatim}

\begin{figure}[H]

{\centering \includegraphics{Lab_2_files/figure-pdf/unnamed-chunk-46-1.pdf}

}

\end{figure}

\hypertarget{lab-2-assignment}{%
\section{\texorpdfstring{\textbf{5.) Lab 2
Assignment}}{5.) Lab 2 Assignment}}\label{lab-2-assignment}}

\textbf{1.} Make a new dataframe called `irisdata' from the `iris' date
built into R.\\
\strut \\
\textbf{2.} Make a histogram of Sepal.Length that compares distributions
for all 3 species in the same graph. Note that color= changes the color
of lines and fill= changes the color of the fill!\\
\strut \\
\textbf{3.)} Make a boxplot that shows how Sepal.Length differs by
Species. Remove the gray background (there are many ways to do that--
any way you want is fine).\\
\strut \\
\textbf{4.)} Make a bar graph that shows Sepal.Length by species. Is
this a good graph or no? Consider the aspects of good vs bad graphs in
the tutorial.\\
\strut \\
\textbf{5.)} Make a scatter plot that shows Sepal.Length by species.
Compare this to your bar graph. Which is more useful and why?\\
\strut \\
\textbf{6.)} Make a line graph comparing Sepal.Length and Sepal.Width by
species. What do you see? This is often the kind of graph we pair with a
linear regression, so thinking about what it shows us is important.\\
\strut \\
\textbf{ALL graphs below should not have a grey background. Use a theme
to remove that}\\
\strut \\
\textbf{7.)} Pick any of your above graphs. Change the colors away from
default to something else. You can either make your own palette or use a
scale\_color\_manual(). Next, do the same using the ggsci package.\\
\strut \\
\textbf{8.)} Next, take the graph from 7 and make each species a
different shape.\\
\strut \\
\textbf{9.)} Take the graph from 8, add a title, change the axes titles,
and make the text larger (I like font size 18).\\
\strut \\
\textbf{10.)} Take the graph from 6 and facet\_wrap() it by species.\\
\strut \\
\textbf{11.)} Using the patchwork package, take any three of your graphs
and panel them so that they all fit together on one page.\\
\strut \\
\textbf{12.)} Render your quarto doc and submit your .html file on
moodle.



\end{document}
